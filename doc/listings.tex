\documentclass{scrreprt}
% \documentclass[12pt]{article}
\usepackage{graphicx}

\usepackage{xr}
\externaldocument{article}

\usepackage[usenames,dvipsnames]{xcolor}
\usepackage{hyperref}
\hypersetup{
        colorlinks,
        linkcolor={red!50!black},
        citecolor={blue!50!black},
        urlcolor={blue!80!black}
}
%
\usepackage{graphicx} %allows for including images
\usepackage{biblatex}

\newcommand{\massa}{{\large \textsc{mass}}}

\makeatletter
\newcommand{\usetocfromothersource}[1]{%
  \begingroup
  \IfFileExists{#1}{%
    \@input{#1}% 
    \@nobreakfalse
  }{}%
  \endgroup
}
\makeatother

\newcommand*{\reff}[1]{%
	{\NoHyper\ref{#1}\endNoHyper}%
	  }

\def\thesection{SI-\arabic{section}}

\begin{document}

\title{Listings of equations, figures, tables and sections of the
article '{\it \bf Musical elements in the discrete-time representation of
sound}' and of the scripts in the MASS toolbox}

\author{\\\\\\ Renato Fabbri (USP),\\
Vilson Vieira da Silva Junior (Cod.ai),\\
Ant\^onio Carlos Silvano Pessotti (Unimep),\\
D\'ebora Cristina Corr\^ea,\\
Osvaldo N. Oliveira Jr. (USP)
}

\maketitle

\begin{abstract}
The article is the main document of the \massa\ toolbox.
Being it of considerable length and complexity,
this document contains listings of its elements to facilitate
its navigation, apprehension and general usage.
\end{abstract}

\tableofcontents

\clearpage
\section{Table of Contents of the article}
\makeatletter
\let\Hy@linktoc\Hy@linktoc@none
\makeatother
\usetocfromothersource{article.toc}

\clearpage
\section{Equations}

\begin{table*}[htp!]
\centering
\caption{Equation numbers and their descriptions.
All these equations are implemented in file \texttt{src/sections/2.py}.}
\begin{tabular}{ c | p{12cm} }
   Number & Description \\\hline
 \reff{eq:dur} & relation between number of samples and duration \\
 \reff{eq:potencia} & power of the wave \\
 \reff{decibels} & decibels by difference by means of the power of each wave \\
 \reff{eq:ampVol} & double amplitude implies $\approx 6dB$ \\
 \reff{eq:potVol} & double power implies $\approx 3dB$ \\
 \reff{eq:dobraVol} & double volume ($10dB$) implies a factor of $\approx 3.16$ for amplitude \\
 \reff{ampDec} & direct relation between variations in amplitude and decibels \\
 \reff{periodicidade} & equivalences in a periodic sound with respect to wavelength, frequency and sample rate \\
 \reff{sinusoid} & sample amplitudes in a sinusoid \\
 \reff{sawTooth} & sample amplitudes in a sawtooth wave \\
 \reff{triangular} & sample amplitudes in a triangular wave \\
 \reff{square} & sample amplitude in a square wave \\
 \reff{sampleandoFormaDeOnda} & samples in a sound derived from a sampled wave period \\
 \reff{recomposicaoFourier} & reconstruction of samples from the Fourier components \\
 \reff{moduloEfase} & reconstruction of real samples (e.g. for audio) from Fourier components \\
 \reff{coefsPareados} & number of pairs of Fourier coefficients which are related to the same frequency \\
 \reff{equivalenciasFreqs} & indexes of equivalent frequencies and coefficients for real signals \\
 \reff{equivalenciasModulos} & equal modules between samples of real signals \\
 \reff{equivalenciasFases} & equivalence in phases between samples of real signals \\
 \reff{eq:reconsCompleta} & complete reconstruction of the signal using Fourier components and previous equations \\
 \reff{eq:notaBasica} & sample sequence related to a basic note \\
 \reff{periodoUnico} & sample sequence of a period of an arbitrary waveform \\
 \reff{eq:notaBasicaTimbre} & note samples derived from a sampled waveform \\
 \reff{eq:distOuvidos} & distance of a (sound) source to each ear given the distance between the ears and an (x,y) position of the source \\
 \reff{eq:dti} & Interaural Time Difference (ITD), the time difference of a sound reaching each ear \\
 \reff{eq:dii} & Interaural Intensity Difference (IID), the intensity difference (in decibels) of a sound reaching each ear \\
 \reff{eq:locImpl} & ITD and IID in terms of sample delays and their amplitudes \\
 \reff{eq:angulo} & azimuthal angle of a (x, y) source \\
 \reff{eq:mixagem} & samples that result from mixing sounds \\
 \reff{eq:concatenacao} & samples that result from concatenating sounds \\
\end{tabular}
\end{table*}

\begin{table*}[htp!]
\centering
\caption{Equation numbers and their descriptions.
All these equations are implemented in file \texttt{src/sections/3.py}.}
\begin{tabular}{ c | p{12cm} }
   Number & Description \\\hline
 \reff{eq:lut} & sample sequence generated by means of a lookup table (LUT) \\
 \reff{freqLinear} & frequency at each sample in a linear transition of frequency \\
 \reff{indiceLinear} & indices of a LUT in a linear transition of frequency \\
 \reff{serieAmostralLin} & sample sequence obtained through a LUT in a linear transition of frequency \\
 \reff{freqExponencial} & frequency at each sample in an exponential transition of frequency (linear pitch) \\
 \reff{indiceExponencial} & indices of a LUT in an exponential transition of frequency (linear pitch) \\
 \reff{serieAmostralLog} & sample sequence obtained through a LUT in an exponential transition of frequency \\
 \reff{seqAmp} & amplitude factors at each sample in an exponential transition of amplitude ($\approx$ linear volume) \\
 \reff{transAmp} & sample sequence with an exponential transition of amplitude ($\approx$ linear volume) \\
 \reff{seqAmpLin} & amplitude factors at each sample in a linear transition of amplitude \\
 \reff{seqAmpDB} & sample sequence in an exponential transition of amplitude ($\approx$ linear volume) with difference given in decibels \\
 \reff{eq:conv} & sample sequence obtained through the convolution of two other sequences (e.g. for applying FIR filters) \\
 \reff{eq:diferencas} & difference equation (e.g. for applying IIR filters) \\
 \reff{eq:passa-baixas} & IIR coefficients for a simple, useful and well-behaved low-pass filter \\
 \reff{eq:passa-altas} & IIR coefficients for a simple, useful and well-behaved high-pass filter \\
 \reff{eq:varAux} & auxiliary variables for the following band-pass and band-reject filters \\
 \reff{eq:passa-banda} & IIR coefficients for a simple, useful and well-behaved band-pass filter \\
 \reff{eq:rejeita-banda} & IIR coefficients for a simple, useful and well-behaved band-reject filter \\
 \reff{eq:branco} & Fourier coefficients of a white noise \\
 \reff{eq:rosa} & Fourier coefficients of a pink noise \\
 \reff{eq:marrom} & Fourier coefficients of a brown noise \\
 \reff{eq:azul} & Fourier coefficients of a blue noise \\
 \reff{eq:violeta} & Fourier coefficients of a violet noise \\
 \reff{eq:preto} & Fourier coefficients of a black noise \\
 \reff{vbrGamma} & indices for a vibrato given its frequency and using a LUT \\
 \reff{vbrAux} & samples for applying a vibrato \\
 \reff{vbrF} & frequency at each sample of a sound with vibrato \\
 \reff{vbrGamma2} & indices for LUT in a sound with vibrato \\
 \reff{vbrT} & sample sequence of a sound with vibrato \\
 \reff{trA} & amplitude at each sample for a tremolo \\
 \reff{trT} & sample sequence of a sound with tremolo \\
 \reff{eq:fmEsp} & components in FM synthesis when both modulator and carrier are sines \\
 \reff{eq:Bessel} & the Bessel function \\
 \reff{eq:specAM} & components in AM synthesis when both modulator and carrier are sines \\
 \reff{fmGammaAux} & indices for LUT in modulator of an FM synthesis \\
 \reff{fmAux} & sample sequence of a modulator in an FM synthesis \\
 \reff{fmF} & frequeny at each sample of a sound derived from FM synthesis \\
 \reff{fmGamma} & indices for the final signal in FM synthesis using LUT \\
 \reff{fmT} & sample sequence of a sound generated through FM and using LUT \\
 \reff{amA} & amplitude at each sample in a sound generated though AM \\
 \reff{amT} & sample sequence of a sound generated through AM and using LUT \\
\end{tabular}
\end{table*}

\begin{table*}[htp!]
\centering
\caption{Equation numbers and their descriptions.
All these equations are implemented in file \texttt{src/sections/3.py}.}
\begin{tabular}{ c | p{12cm} }
   Number & Description \\\hline
 \reff{eq:vinculos} & an example of bonds between musical characteristics \\
 \reff{eq:fDoppler} & relation between frequencies and speed in the Doppler effect \\
 \reff{eq:aDoppler} & relation between position, speed and amplitude in the Doppler effect \\
 \reff{eq:ffDoppler} & relation between position, speed and amplitude in the Doppler effect \\
 \reff{eq:p1rev} & samples of a FIR filter for the first period of a reverberation \\
 \reff{eq:p2rev} & samples of a FIR filter for the second period of a reverberation \\
 \reff{eq:rev} & samples of the FIR filter for a reverberation (considering both first and second periods) \\
 \reff{eq:adsr} & amplitude factors for each sample in an ADSR envelope \\
 \reff{eq:adsrApl} & sample sequence of a sound with an ADSR envelope \\
\end{tabular}
\end{table*}

\begin{table*}[htp!]
\centering
\caption{Equation numbers and their descriptions.
All these equations are implemented in file \texttt{src/sections/4.py}.}
\begin{tabular}{ c | p{12cm} }
   Number & Description \\\hline
 \reff{eq:micro} & pitches represented in two divisions of the octave \\
 \reff{escSim} & perfectly symmetric scales in each octave with the twelve semitones \\
 \reff{eq:escalas} & diatonic scales \\
 \reff{eq:relacaoDia} & the succession of tones and semitones of a diatonic scale \\
 \reff{eq:escalasMenores} & sequences of semitones for the three types minor scales \\
 \reff{eq:serieHarmonica} & harmonic series in terms of semitones \\
 \reff{triades} & triads (chords constituted by thirds) \\
 \reff{eq:rhythmicUnit} & a convention to specify a unit of rhythmic division or agglomeration \\
 \reff{eq:groups} & definition of algebraic groups \\
\end{tabular}
\end{table*}


\clearpage
\section{Figures}\label{sec:lfigures}

\begin{table*}[htp!]
\centering
\caption{Figure numbers and their descriptions.
All these equations are implemented in files \texttt{src/aux/*}.}
\begin{tabular}{ c | p{12cm} }
   Number & Description \\\hline
 \reff{fig:PCM} & PCM PCM audio (discrete and digital) samples \\
 \reff{fig:formasDeOnda} & synthetic and sampled waveforms \\
 \reff{fig:espectroDeOndas} & spectrum of basic waveforms \\
 \reff{fig:espectroOboe} & spectrum of a real note and of one derived from one sampled period \\
 \reff{fig:amostras2} & sinusoid represented by two samples \\
 \reff{fig:amostras3} & sinusoid represented by three samples \\
 \reff{fig:amostras4} & sinusoids in 4 samples \\
 \reff{fig:formas4} & basic waveforms within 4 samples \\
 \reff{fig:amostras6} & sinusoids in 6 samples \\
 \reff{fig:formas6} & basic waveforms withing 6 samples \\
 \reff{fig:spac} & ITD and IID (spatial localization cues)\\
 \reff{fig:mixagem} & mixing PCM audio \\
 \reff{fig:concatenacao} & concatenation/juxtaposition of PCM audio \\
 \reff{fig:lut} & Lookup table \\
 \reff{fig:transicao} & transitions of intensity \\
 \reff{fig:conv} & convolution \\
 \reff{fig:delays} & convolution yielding time shifting, multiple time delays, sound amalgam \\
 \reff{fig:iir} & frequency response of useful IIR filters \\
 \reff{fig:ruidos} & spectrum and waveform of noises \\
 \reff{fig:vibrato} & spectrogram of a vibrato \\
 \reff{fig:tremolo} & waveform of a tremolo \\
 \reff{fig:adsr} & ADSR envelope \\
 \reff{fig:movContraponto} & counterpoint movements \\
 \reff{fig:pulsoSubAgl} & musical metric in terms of divisions of temporal units \\
 \reff{fig:climax} & distinctions of musical climax by localization \\
\end{tabular}
\end{table*}

\clearpage
\section{Tables}

\begin{table*}[htp!]
\centering
\caption{Table numbers and their descriptions.}
\begin{tabular}{ c | p{12cm} }
   Number & Description \\\hline
 \reff{eq:intervalos} & musical intervals, their notations and qualities \\
 \reff{tab:harmonia} & tonal harmonic functions in the major scale \\
 \reff{tab:duracoes} & duration scales yielding perception of pitch and rhythm \\
 \reff{tab:change} & music by permutation of units (change ringing) \\
\end{tabular}
\end{table*}

\clearpage
\section{Scripts}
\subsection{For all equations and relations in each section.}

\begin{table*}[htp!]
\centering
\caption{Script files and their descriptions.}
\begin{tabular}{ c | p{12cm} }
   Filename & Description \\\hline
  \texttt{src/sections/2.py} & implementation of all the equations for the basic note in PCM audio in Section~\ref{sec:discNote} \\
  \texttt{src/sections/3.py} & implementation of all the equations for variations within a note described in Section~\ref{sec:varInternas} \\
  \texttt{src/sections/4.py} & implementation of all the techniques for assembling notes into music described in Section~\ref{sec:notesMusic} \\
\end{tabular}
\end{table*}

\clearpage
\subsection{To render musical pieces}

\begin{table*}[htp!]
\centering
\caption{Piece names, script files and the concepts they exemplify from Section~\ref{sec:discNote}. All files are in the directory \texttt{src/pieces2/}.}
\begin{tabular}{ p{5cm} | p{3.5cm} | p{6.2cm} }
   Name & Filename & Description \\\hline
 Reduced-fi & \texttt{reduced-fi.py} & concatenation of simple notes \\
 Sonic pictures & \texttt{quadrosSonoros.py} & mixing of simple notes  \\
\end{tabular}
\end{table*}

\begin{table*}[htp!]
\centering
\caption{Piece names, script files and the concepts they exemplify from Section~\ref{sec:internalVar}. All files are in the directory \texttt{src/pieces3/}.}
\begin{tabular}{ p{5cm} | p{3.5cm} | p{6.2cm} }
   Name & Filename & Description \\\hline
 ADa and SaRah & \texttt{ADAandSaRah.py} & ADSR envelope \\
 Tremolos, Vibratos and Frequency & \texttt{bonds.py} & bonds between tremolos, vibratos and frequency  \\
 Bella Rugosi & \texttt{bellaRugosi.py} & rugosity achieved through frequencies between 15 adn 30 Hz \\
 Children Choir & \texttt{childChoir.py} & achieving choir sonorities by small variations of the same basic note \\
 Noisy Band & \texttt{noisyBand.py} & using various noises \\
 ParaMeter Transitions & \texttt{paraMeter.py} & gradual changes of parameters within one note \\
 Little Train of Impulsive Hillbilies~\footnote{This names translates into ``Trenzinho de Caipiras Impulsivos'' which is a pun with the Villa-Lobos' ``Trenzinho do Capira''.} & \texttt{littleTrain.py} & the use of convolution with impulses to achieve rhythm \\
 Shakes and Wiggles & \texttt{shakesWiggles.py} &  tremolos and vibratos \\
\end{tabular}
\end{table*}

\begin{table*}[htp!]
\centering
\caption{Piece names, script files and the concepts they exemplify from Section~\ref{sec:notesMusic}. All files are in the directory \texttt{src/pieces4/}.}
\begin{tabular}{ p{5cm} | p{3.5cm} | p{6.2cm} }
   Name & Filename & Description \\\hline
 Acorde Cedo & \texttt{acordeCedo.py} & chord successions and modulation \\
 Conta Ponto & \texttt{contaPonto.py} & melodic lines conducted within the rules of counterpoint \\
 Crystals & \texttt{crystals.py} & symmetric musical scales \\
 Dirracional & \texttt{dirracional.py} & directional arcs \\
 Intervals & \texttt{intervals.py} & musical intervals \\
 MicroTone & \texttt{microTone.py} & microtonality (the use of intervals smaller than the semitone) \\
 Poly-Hit-My & \texttt{polyHitMy.py} & polyrhythm (multiple metrics at once) \\
\end{tabular}
\end{table*}

\clearpage
\subsection{To render the figures used in the article}
The files \texttt{src/aux/*} render each of the figures (listed above in Section~\ref{sec:lfigures}.

\clearpage
\section{Other documents}
The files in \texttt{latex/*} render the article PDF and this Supporting Information file.

The script \texttt{src/aux/iso226.py} can generate the Equal Loudness Contour as described in its latest revision~\cite{iso226}.

The article is largely based on the MSc dissertation~\cite{dissertacao}.

\end{document}
