\documentclass{article}
\usepackage[utf8]{inputenc}
\usepackage{graphicx}
\usepackage{listings}

\usepackage[T1]{fontenc}
\usepackage[scaled]{beramono}
% \renewcommand*\familydefault{\ttdefault}
\usepackage{listings}
\usepackage{url}
\usepackage{color}
\usepackage[usenames,dvipsnames,svgnames,table]{xcolor}

\lstset{
  extendedchars=\true,
  inputencoding=utf8x,
  language=Python,
  %showstringspaces=false,
  formfeed=\newpage,
  tabsize=4,
  %commentstyle=\itshape,
  basicstyle=\ttfamily,
  basicstyle={\small\fontfamily{fvm}\fontseries{m}\selectfont},
  lineskip={-1.5pt},
  belowskip=0pt,
  aboveskip=0pt,
  commentstyle=\color{Apricot}\bfseries,
  %commentstyle=\color{red}\itshape,
  stringstyle=\color{red},
  identifierstyle=\color{PineGreen},
  showstringspaces=false,
  keywordstyle=\color{blue}\bfseries,
  moredelim=[il][\large\textbf]{\#\# },
  morekeywords={models,range},
  numbers=left,
  numberstyle=\tiny,%\color{blue}\bfseries,
  literate=%
  {ã}{{\~a}}1
  {â}{{\^a}}1
  {õ}{{\~o}}1
  {á}{{\'a}}1
  {ú}{{\'u}}1
  {í}{{\'i}}1
  {é}{{\'e}}1
  {Ç}{{\c{C}}}1
  {Õ}{{\~O}}1
  {Ê}{{\^E}}1
  {ó}{{\'o}}1
  {à}{{\`a}}1
  {Â}{{\^A}}1
  {ô}{{\^o}}1
  {ê}{{\^e}}1
  {ç}{{\c{c}}}1
}
 
\newcommand{\code}[2]{
% \hrulefill[*]
%  \bigskip % 1
%  \centerline{********} % 1

%\sepline



% \sepstars

 \subsubsection*{#1}
 \lstinputlisting{#2}
}

\begin{document}

\title{Code in the MASS framework}
\author{\\\\\\ Renato Fabbri (USP),\\
Vilson Vieira da Silva Junior (Cod.ai),\\
Ant\^onio Carlos Silvano Pessotti (Unimep),\\
D\'ebora Cristina Corr\^ea,\\
Osvaldo N. Oliveira Jr. (USP)
}

\maketitle

\begin{abstract}
This document displays the Python code in the MASS
framework.
The code is accessible as Python scripts in the main repository~\cite{massRepo},
and this PDF is made available because it might facilitate
browsing the implementations.
Check the final consideration in this document for further directions.
\end{abstract}

\section{Sections}
Here is the code related to each section of the MASS article~\cite{massArticle}.

% \lstinputlisting[language=Python]{../src/sections/2.py}
\code{Python implementation of equations in Section 2}{../src/sections/2.py}

\clearpage
\code{Python implementation of equations in Section 3}{../src/sections/3.py}

\clearpage
\code{Python implementation of equations in Section 4}{../src/sections/4.py}

\clearpage
\section{Musical pieces}
The code for the musical pieces are omitted from this PDF because
it would make the document lengthy.
Please see~\cite{massListings} and~\cite{massRepo}
to know what are the available scripts for rendering musical pieces
and reach them.

\section{Auxiliary files}
The code for the auxiliary files (e.g. to render the figures in the article~\cite{massArticle}
are omitted from this PDF because
it would make the document lengthy.
Please see~\cite{massListings} and~\cite{massRepo}
to know what are the available auxiliary scripts
and reach them.

\section{Final considerations}
This document exhibits the code that implements the relations
in the sections of~\cite{massArticle}.
All this scripts are available in~\cite{massRepo} with
other documentations and further scripts e.g. to render musical
pieces and the figures in~\cite{massArticle}.
One should also reach~\cite{massListings} to know
about the resources in the MASS framework.
This document should be available at~\cite{massCode}.

\begin{thebibliography}{9}
    \bibitem{massRepo}
    Fabbri (2017). M\'usica no \'audio digital : descri\c{c}\~ao psicof\'isica e caixa de ferramentas. MSc dissertation. Available at: \url{http://www.teses.usp.br/teses/disponiveis/76/76132/tde-19042013-095445/publico/RenatoFabbri_ME_corrigida.pdf}

\bibitem{massArticle}
    ISO: 226. (2003). Normal Equal-Loudness Level Contours.

\bibitem{massListings}
    ISO: 226. (2003). Normal Equal-Loudness Level Contours.

\bibitem{massCode}
    ISO: 226. (2003). Normal Equal-Loudness Level Contours.
\end{thebibliography}

\end{document}
